\chapter{Conclusions}
Ens vam decidir pel treball de càlcul de trajectòries interplanetaries des del primer moment degut a que és un tema que realment ens apassiona. Vivim en una època en què l'arribada a Mart és imminent. Això ens anima a pensar que encara es pot anar més lluny, però per això cal saber com arribar-hi.

No obstant, el càlcul de trajectòries interplanetàries no és una tasca fàcil, per això s'empren diverses aproximacions. El mètode de \textit{Patched Conics Approximation} és  molt útil per a obtenir una primera aproximació de l'òrbita que ha de seguir una sonda per viatjar d'un planeta a un altre. Tanmateix, els resultats no són exactes, ja que es tracta d'un mètode numèric i es basa en aproximacions.

Com es pot veure en els resultats obtinguts, alguns valors es desvien de la solució aportada pel professor. Aquesta variació s'explica perquè des dels primers càlculs els resultats que s'obtenen no són 100\% iguals als de referència. És a dir, s'observa que una petita variació en els vectors de posició comporta una gran variació en els angles d'Euler. No obstant, es pot afirmar que els resultats que s'han obtingut són satisfactoris ja que, tot i no ser exactes, són coherents i s'aproximen a la solució real.

En quant a la distribució del treball, aquest s'ha dividit en petites funcions que s'han repartit entre els diferents membres de l'equip. Un cop elaborada, cada funció s'ha intentat validar de forma independent a la resta del codi. D'aquesta forma és més fàcil trobar possibles errors i s'assegura que el programa global és funcional i robust.

Finalment, a fi de ser el més eficients possible, s'ha emprat un sistema de control de versions \textit{Git}, per així poder treballar simultàniament sobre els mateixos arxius sense perdre cap informació.

Així doncs, s'ha aconseguit combinar la nostra passió pel tema amb una bona organització. Això ha ajudat a que poguéssim gaudir alhora que apreníem molt amb el tema escollit.
