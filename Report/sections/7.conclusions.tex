\chapter{Conclusions}
Ens vam decidir pel treball del càlcul de trajectòries interplanetaries des del primer moment degut a que és un tema que realment ens apassiona. Estem vivint una època en que l'arribada a Mart és imminent. Això ens anima a pensar que encara és pot anar més lluny.

El mètode del patched conics....molt útil per una primer approx...bla bla-....els resultats s'han de refinar amb un mètode numèric....blablabla....alguns valros poden diferir de la solució del professor pel fer que una petita variació en els vectors orbitals comporta una gran variació dels àngles d'euler...blabla

EL treball s'ha dividit en petites funcions que a continuació s'han repartit entre els membres de l'equip. Cada funció s'ha intentat validar de forma independent a la resta del codi, per així poder obtenir un programa global funcional i robust.

Per aspectes d'eficiència s'ha emprat un sistema de control de versions git, per així poder treballar simultàniament sobre els mateixos arxius sense perdre cap informació.

Així doncs, s'ha aconseguit combinar la nostra passió pel tema, amb una bona organització. Això ha ajudat a que poguem gaudir i aprendre molt amb el tema escollit..
