\chapter{Codi}
Com s'ha anat comentant, el codi està dividit en funcions amb una tasca definida i clara. A continuació s'expliquen els codis emprats com a funcions auxiliars pels altres codis ja vists.
\section{Main}
\lstinputlisting[language=Matlab]{../../main.m}
% \section{Transfer Orbit}
% \lstinputlisting[language=Matlab]{../../orbita_interplanetaria.m}
% \lstinputlisting[language=Matlab]{../../hyperbolic_orbit.m}
\section{outHyperbola}
\label{outHyperbola}
\lstinputlisting[language=Matlab]{../../outHyperbola.m}
\section{OrbitalVectors}
\label{OrbitalVectors}
\lstinputlisting[language=Matlab]{../../OrbitalVectors.m}
\section{JulianDate}
S'aplica l'algoritme de conversió vist a clase. \cite[pàg.48]{Calaf2017a}
\lstinputlisting[language=Matlab]{../../JulianDate.m}
\section{checkTangent}
Aquesta funció simplement s'encarrega de rebre un quocient, i segons el signe, decidir en quin quadrant situar l'angle que surt de fer l'arc tangent.
\lstinputlisting[language=Matlab]{../../checkTangent.m}
